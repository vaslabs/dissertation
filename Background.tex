Morphogenesis is defined by Alan Turing as the process of how cells of an embryo create shape out of an initially chaotic behaviour \cite{turing_chemical_1990}. Cells are interacting with each other by diffusing chemical substances which in Turing's morphogenesis model are defined as morphogens. The term morphogen is abstract and should not be confused with DNA genes, chromosomes or certain chemical elements. 
A morphogen is considered to be a chemical substance that has some theoretical properties which give to cell specific characteristics. Morphogens may also react with each other in the same way chemical molecules do. Reactions and diffusion happen over time and thus morphogenesis process can be a dynamical system.   
Thus, it can be modelled mathematically using differential equations, by suggesting that each cell has a state according to its morphogen concentrations and that state is subject to change over time.

The approach followed for simulation and analysis of morphogenesis' models uses a variety of concepts of the mathematical, biochemical and programming domain. Such concepts are discussed in this report providing the knowledge basis that is necessary for the reader to follow the approach and the implementation procedure of the project.

Finally, various papers and articles \cite{garvie_methodology}\cite{heineike_modeling_2002}\cite{kassam_solving_2003} that study the phenomenon of morphogenesis and pattern generation were used to identify mathematical models of the reaction-diffusion process and were the spark for further exploration of ideas related to morphogenesis. 




