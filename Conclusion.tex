\chapter{Conclusion}
The project generated animations and movies that visualise how cells can form complex structures. This was done with the use of two mathematical models, the Gray-Scott model and the L-Systems equations. A third model, the  Gizburg-Landau was used to explore the possibility of sound generation. The ordinary differential equations of those models, were integrated with the ODE solvers of matlab and matrix operations. The results were shown as images by using conversion algorithms to normalise chemical concentrations into ranges of colour values. Results were shown in the form of plotted graphs as well.

Further goals of the project were to exploit the idea of morphogenesis to research other concepts. One was to generate sound instead of images. Experimentation with sound led to the definition of ideas and implementation of tools which may be used in the future to explore the possibility of audio production by the use of models of morphogenesis. The graph results of the project in terms of audio production present similarities to the sound waves. Thus, further analysis and exploration of the concept is encouraged.

Next, the project studied how mutations affect the state of a system. The experiments take a finished structure of a system and alter the morphogen concentrations of some cells with a given probability. Then, the system is integrated again. It has been shown that with a very low probability (0.001) that affects only a tiny amount of cells, the system recovers the structure. On the other hand, a probability of 0.01 is able to alter the structure of a system significantly.

The latter experiment arises further questions or ideas on how to use mutation to study morphogenesis. What will happen if the mutation is done according to the values of a different system, in the same way that grafting is done for plants? What would be the outcome if the mutation changes the diffusion coefficients as well? Is there a way to recover the system to regain its pre-mutant structure? 

The last goal was to find a way to apply the concepts of morphogenesis in a computer science problem. The `diffusion-inspired algorithm' was to create a process scheduler that uses diffusion to give each process different time slices on which processes are allowed to be executed. Although the algorithm has a high computational complexity it seems to be more effective than a random scheduler or a priority scheduler. The difference between the diffusion-inspired scheduler and the Round Robin is very small. For some parameters the diffusion-inspired scheduler even seems to have less inactivity and less total time than the Round Robin. 

There are a lot of parameters in the scheduling problem that have not been addressed. The diffusion-inspired scheduler was not compared to itself with different parameters. In addition, schedulers may be tested with processes that have high or low rates in requesting input/output time. Thus, exploring on which environments each scheduler is best, or which parameters make the diffusion-inspired scheduler better in different environments is suggested. The software framework that was developed provides easy to maintain and expand code motivating further experiments.

The project combined a mixture of biology, chemistry, mathematics, engineering and computer science. Time limited further investigation of more conepts and ideas. The conclusion of all this work, is that morphogenesis is not just a mathematical model that generates patterns. It is a framework enabling the testing of physical systems, which has the potential to give new ideas and applications for researching or experimenting with concepts that reflect in the real life world.  
